\documentclass[russian,utf8,simple,emptystyle]{eskdtext}
\usepackage[numberright]{eskdplain}
\usepackage{paratype}
\usepackage{longtable}
\usepackage{array}
\usepackage{epstopdf}
\usepackage{multirow}
%\usepackage[warn]{mathtext} 
\usepackage{amssymb,amsmath,amsthm,latexsym}
\usepackage[usenames,dvipsnames]{color}
\usepackage{listings}

\ESKDdepartment{ФЕДЕРАЛЬНОЕ АГЕНТСТВО ПО ОБРАЗОВАНИЮ РФ}
\ESKDcompany{МГТУ им. Н.Э.Баумана}
\ESKDclassCode{23 0102}
\ESKDtitle{Курсовая работа по дисциплине}
\ESKDdocName{<<Сетевые технологии>>}
\ESKDsignature{<<Локальная безадаптерная сеть>>}
\ESKDauthor{Гуща~А.~В}
\ESKDauthor{Нардид~А.~Н}
\ESKDauthor{Оганян~Л.~П}
\ESKDchecker{Галкин~В.~А}
\ESKDtitleApprovedBy{\hspace{0cm}}{Галкин~В.~А}
\ESKDtitleAgreedBy{\hspace{0cm}}{Галкин~В.~А}
\ESKDtitleDesignedBy{Студент 3 курса группы ИУ5-72}{Гуща~А.~В}
\ESKDtitleDesignedBy{Студент 3 курса группы ИУ5-72}{Нардид~А.~Н}
\ESKDtitleDesignedBy{Студент 3 курса группы ИУ5-72}{Оганян~Л.~П}

\renewcommand{\ESKDtheTitleFieldX}{%
Москва
 
\ESKDtheYear~г.}

\renewcommand{\baselinestretch}{1}
\renewcommand{\arraystretch}{1.5}

\makeatletter
\newcommand\FontSizesXI{%
\renewcommand\normalsize{%
   \@setfontsize\normalsize\@xipt{13.6}%
   \abovedisplayskip 11\p@ \@plus3\p@ \@minus6\p@
   \abovedisplayshortskip \z@ \@plus3\p@
   \belowdisplayshortskip 6.5\p@ \@plus3.5\p@ \@minus3\p@
   \belowdisplayskip \abovedisplayskip
   \let\@listi\@listI}
\normalsize
\renewcommand\small{%
   \@setfontsize\small\@xpt\@xiipt
   \abovedisplayskip 10\p@ \@plus2\p@ \@minus5\p@
   \abovedisplayshortskip \z@ \@plus3\p@
   \belowdisplayshortskip 6\p@ \@plus3\p@ \@minus3\p@
   \def\@listi{\leftmargin\leftmargini
               \topsep 6\p@ \@plus2\p@ \@minus2\p@
               \parsep 3\p@ \@plus2\p@ \@minus\p@
               \itemsep \parsep}%
   \belowdisplayskip \abovedisplayskip
}
\renewcommand\footnotesize{%
   \@setfontsize\footnotesize\@ixpt{11}%
   \abovedisplayskip 8\p@ \@plus2\p@ \@minus4\p@
   \abovedisplayshortskip \z@ \@plus\p@
   \belowdisplayshortskip 4\p@ \@plus2\p@ \@minus2\p@
   \def\@listi{\leftmargin\leftmargini
               \topsep 4\p@ \@plus2\p@ \@minus2\p@
               \parsep 2\p@ \@plus\p@ \@minus\p@
               \itemsep \parsep}%
   \belowdisplayskip \abovedisplayskip
}
\renewcommand\scriptsize{\@setfontsize\scriptsize\@viiipt{9.5}}
\renewcommand\tiny{\@setfontsize\tiny\@vipt\@viipt}
\renewcommand\large{\@setfontsize\large\@xiipt{14}}
\renewcommand\Large{\@setfontsize\Large\@xivpt{18}}
\renewcommand\LARGE{\@setfontsize\LARGE\@xviipt{22}}
\renewcommand\huge{\@setfontsize\huge\@xxpt{25}}
\renewcommand\Huge{\@setfontsize\Huge\@xxvpt{30}}
}
\makeatother

\lstset{ %
  backgroundcolor=\color{white},   % choose the background color; you must add \usepackage{color} or \usepackage{xcolor}
  basicstyle=\footnotesize,        % the size of the fonts that are used for the code
  breakatwhitespace=true,          % sets if automatic breaks should only happen at whitespace
  breaklines=true,                 % sets automatic line breaking
  captionpos=b,                    % sets the caption-position to bottom
  commentstyle=\color{ForestGreen},% comment style
  deletekeywords={...},            % if you want to delete keywords from the given language
  escapeinside={\%*}{*)},          % if you want to add LaTeX within your code
  extendedchars=true,              % lets you use non-ASCII characters; for 8-bits encodings only, does not work with UTF-8
  frame=none,                      % adds a frame around the code
  keepspaces=true,                 % keeps spaces in text, useful for keeping indentation of code (possibly needs columns=flexible)
  keywordstyle=\color{BrickRed},   % keyword style
  language=Haskell,                % the language of the code
  morekeywords={*,...},            % if you want to add more keywords to the set
  numbers=left,                    % where to put the line-numbers; possible values are (none, left, right)
  numbersep=5pt,                   % how far the line-numbers are from the code
  numberstyle=\scriptsize\color{CadetBlue}, % the style that is used for the line-numbers
  rulecolor=\color{black},         % if not set, the frame-color may be changed on line-breaks within not-black text (e.g. comments (green here))
  showspaces=false,                % show spaces everywhere adding particular underscores; it overrides 'showstringspaces'
  showstringspaces=false,          % underline spaces within strings only
  showtabs=false,                  % show tabs within strings adding particular underscores
  stepnumber=1,                    % the step between two line-numbers. If it's 1, each line will be numbered
  stringstyle=\color{OliveGreen},  % string literal style
  tabsize=2,                       % sets default tabsize to 2 spaces
  title=\lstname                   % show the filename of files included with \lstinputlisting; also try caption instead of title
}

\begin{document}
\maketitle
\tableofcontents
\newpage
\section{Введение}

Данная программа, <<PowerCom>>, выполненная в рамках курсовой работы по предмету <<Сетевые технологии>>, предназначена для организации обмена текстовыми сообщениями мужду соединенными с помощью интерфейса RS-232C компьютерами. Программа позволяет обмениваться с двум компьютерам, соединенный через COM-порты, текстовыми сообщениями, при условии запуска этой программы на обоих компьютерах.

\section{Требования к программе}
К программе предъявляются следующие требования. Программа должна:
\begin{itemize}
\item Устанавливать соединение между компьютерами и контролировать его целостность;
\item Обеспечивать правильность передачи и приема данных с помощью алгоритма циклического кодирования пакета;
\item Обеспечивать функцию передачи сообщений;
\item Программа должна выполняться под управлением операционной системы OS Windows XP/7.
\end{itemize}

\section{Определение структуры программного продукта}
При взаимодействии компьютеров между собой выделяются несколько уровней: нижний уровень должен обеспечивать соединение компьютера со средой передачи, а верхний - обеспечить интерфейс пользователя. Программа разбивается на три уровня: физический, канальный и прикладной (см. Приложение <<Структурная схема программы>>).
\begin{itemize}
\item Физический уровень предназначен для сопряжения компьютера со средой передачи;
\item Канальный уровень занимается установлением и поддержанием соединения, формированием и проверкой пакетов обмена протоколов верхний модулей;
\item Прикладной уровень занимается выполнением задач программы.
\end{itemize}

\section{Физический уровень}
\subsection{Функции физического уровня}
Основными функциями физического уровня являются:
\begin{enumerate}
\item Задание параметров COM-порта;
\item Установление физического канала;
\item Разъединение физического канала;
\item Передача информации из буфера в интерфейс;
\item Прием информации и ее накопление в буфере.
\end{enumerate}

\subsection{Описание физического уровня}
Последовательная передача данных означает, что данные передаются по единственной линии. При этом биты байта данных передаются по очереди с использованием одного провода. Для синхронизации группе битов данных обычно предшествует специальный \textit{стартовый бит}, после группы битов следуют \textit{бит проверки на четность} и один или два \textit{стоповых бита} (см. рисунок~\ref{fig:phys-level}. Иногда бит проверки на четность может отсутствовать.

\begin{figure}[h!]
\centering
\includegraphics[scale=1.0]{phys-level}
\caption{Временная диаграмма передачи кадра.}
\label{fig:phys-level}
\end{figure}

Из рисунка видно, что исходное состояние линии последовательной передачи данных - уровень логической единицы. Это состояние линии называют отмеченным - \textbf{MARK}. Когда начинается передача данных, уровень линни переходит в логический нуль. Это состояние линии называют пустым - \textbf{SPACE}. Если линия находится в таком состоянии больше определенного времени, считается, что линия перешла в состояние разрыва связи - \textbf{BREAK}.

Стартовый бит \textbf{START} сигнализирует о начале передачи данных. Далее передаются биты данных, вначале младшие, затем старшие.

Контрольный бит формируется на основе правила, которое создается при настройке передающего и принимающего устройства. Контрольный бит может быть установлен с контролем на четность, нечетность, иметь постоянное постоянное значение логической единицы, либо отсутствовать совсем.

Если используется бит четности \textbf{P}, то передается и он. Бит четности имеет такое значение, чтобы в пакете битов общее количество единиц  (или нулей) было четно или нечетно, в зависимости от установки регистров порта. Этот бит служит для обнаружения ошибок, которые могут возникнуть при передаче данных из-за помех на линнии. Приемное устройство заново вычисляет четность данных и сравнивает результат с принятым битом четности. Если четность не совпала, то считается, что данные переданы с ошибокй. Конечно, такой алгоритм не дает стопроцентной гарантии обнаружения ошибок. Так, если при передачи данных изменилось четное число битов, то четность сохраняется, и ошибка не будет обнаружена. Поэтому на практике применяются более сложные методы обнаружения ошибок.

В самом конце передаются один или два стоповых бита \textbf{STOP}, завершающих передачу байта. Затем до прихода следующего стартового бита линия снова переходит в состояние \textbf{MARK}.

Использование бита четности, стартовых и стоповых битов определяют формат передачи данных. Очевидно, что передатчик и приемник должны использовать один и тот же формат данных, иначе обмен будет невозможен.

Другая важная характеристика - скорость передачи данных. Она также должна быть одинаковой для передатчика и приемника. Скорость передачи данных обычно измеряется в бодах. Иногда используется другой термин - биты в секунду (bps). Здесь имеется в виду эффективная скорость передачи данных, без учета служебных битов.

Интерфейс RS-232C описывает несимметричный интерфейс, работающий в режиме последовательного обмена двоичными данными. Интерфейс поддерживает как асинхронный, так и синхронный режимы работы.

Последовательная передача данных означает, что данные передаются по единственной линии. При этом биты байта данных передаются по очереди с использованием одного провода. Интерфейс называется несимметричным, если для всех цепей обмена интерфейса используется один общий возвратный провод - сигнальная <<земля>>.

\begin{center}
\begin{table}[h!]
\begin{tabular}{>{\centering}p{1.4cm}|c|c|>{\centering}p{1.3cm}}
Номер контакта & Обозначение & Назначение & Цепь 
\tabularnewline
\hline
1 & DCD (Data Carrier Detect) & Обнаружение несущей & 109 \tabularnewline
2 & RD (Receive Data)  & Принимаемые данные  & 104 \tabularnewline
3 & TD (Transmit Data) & Отправляемые данные & 103 \tabularnewline
4 & DTR (Data Terminal Ready) & Готовность терминала к работе & 108/2 \tabularnewline
5 & SG (Signal Ground) & Земля сигнала (схемная) & 102 \tabularnewline
6 & DSR (Data Set Ready) & Готовность DCE          & 107 \tabularnewline
7 & RTS (Request To Send) & Запрос передачи         & 105 \tabularnewline
8 & CTS (Clear To Send)  & Готовность DCE к приему     & 106 \tabularnewline
9 & RI (Ring Indicator)  & Индикатор вызова        & 125
\end{tabular}
\caption{Интерфейс девяти контактного разъема.}
\label{table:phys-interface}
\end{table}
\end{center}

В интерфейсе реализован биполярный потенциальный код на линиях между DTE и DCE. Напряжения сигналов в цепях обмена симметричны по отношению к уровню сигнальной <<земли>> и составляют не менее +3В для двоичного нуля и не более -3В для двоичной единицы.

Каждый байт данных сопровождается специальными сигнальными сигналами <<старт>> - стартовый бит и <<стоп>> - стоповый бит. Сигнал <<старт>> имеет продолжительность в один тактовый интервал, а сигнал <<стоп>> может длиться один, полтора или два такта.

При синхронной передаче данных через интерфейс передаются сигналы синхронизации, без которых компьютер не может правильно интерпретировать потенциальный код, поступающий по линии RD.

\subsection{Нуль-модемный интерфейс}
Обмен сигналами между адаптером компьютера и модемом (или вторым компьютером присоединенном к исходному посредством кабеля стандарта RS-232C) строится по стандартному сценарию, в котором каждый сигнал генерируется сторонами лишь после наступления определенных условий. Такая процедура обмена информацие называется запрос/ответным режимом или \textbf{<<рукопожатием>>} (\textbf{handshaking}). Большинство из приведенных в таблице сигналов как раз и нужны для аппаратной реализации <<рукопожатия>> между адаптером и модемом.

Обмен сигналами между сторонами интерфейса RS-232C выглядит так:
\begin{enumerate}
\item Компьютер после включения питания выставляет сигнал \textbf{DTR}, который постоянно удерживается активным. Если модем включен в электросеть и исправен, он отвечает компьютеру сигналом \textbf{DSR}. Этот сигнал служит подтверждением тог, что \textbf{DTR} принят, и информирует компьютер о готовности модема к приему информации;
\item Если компьютер получил сигнал \textbf{DSR} и хочет передать данные, он выставляет сигнал \textbf{RTS};
\item Если модем готов принимать данные, он отвечает сигналом \textbf{CTS}. Он служит для компьютера подтверждением того, что \textbf{RTS} получен модемом и модем готов принять данные от компьютера. С этого момента адаптер может бит за битом передавать информацию по линии \textbf{TD};
\item Получив байт данных, модем может сбросить свой сигнал \textbf{CTS}, информируя компьютер о необходимости <<притормозить>> передачу следующего байта, например, из-за переполнения внутреннего буфера; программа компьютера, обнаружив сброс \textbf{CTS}, прекращает передачу данных, ожидая повторного появления \textbf{CTS}.
\end{enumerate}

Когда модему необходимо передать данные в компьютер, модем выставляет сигнал  \textbf{DCD}. Программа компьютера, принимающая данные, обнаружив этот сигнал, читает приемный регистр, в который сдвиговый регистр <<собрал>> биты, принятые по линии приема данных \textbf{RD}. Когда для связи используются только приведенные в таблице данные, компьютер не может попросить модем <<повременить>> c передачей следующего байта. Как следствие, существует опсность переполнения помещенного ранее в приемном регистре байта данных вновь <<собранным>> байтом. Поэтому при приеме информации компьютер должен очень быстро освобождать приемный регистр адаптера. В полном наборе сигналов RS-232C есть линии, которые могут аппаратно <<приостановить>> модем.

Нуль-модемный интерфейс характерен для прямой связи компьютеров на небольшом расстоянии (длина кабеля до 15 метров). Для нормальной работы двух непосредственно соединенных компьютеров нуль-модемный кабель должен выполнять следующие соединения:
\begin{enumerate}
\item RI+DSR-1 $\leftrightarrow$ DTR-2;
\item DTR-1 $\leftrightarrow$ RI-2+DSR-2;
\item CD-1 $\leftrightarrow$ CTS-2 + RTS-2;
\item CTS-1+RTS-1 $\leftrightarrow$ CD-2;
\item RD-1 $\leftrightarrow$ TD-1;
\item TD-1 $\leftrightarrow$ RD-1;
\item SG-1 $\leftrightarrow$ SG-2;
\end{enumerate}
Знак <<+>> означает соединение соответствующих контактов на одной стороне кабеля.

\section{Настройка COM-порта средствами Haskell}
Язык Haskell имеет все нобходимые средства для работы с COM портами. Для этого необходимо установить пакет \textbf{serialport } через пакетный менеджер \textbf{cabal}. 

\subsection{Описание типов данных}
\begin{itemize}
\item \textbf{CommSpeed} - скорость передачи данных в бодах в секунду.
\begin{lstlisting}
data CommSpeed =
	CS110   |
	CS300   |
	CS600   |
	CS1200  |
	CS2400  |
	CS4800  |
	CS9600  |
	CS19200 |
	CS38400 |
	CS57600 |
	CS57600 
\end{lstlisting}
Определенные instances:
\begin{lstlisting}
Show CommSpeed
\end{lstlisting}

\item \textbf{StopBits} - количество используемых стоп битов.
\begin{lstlisting}
data StopBits =
	One |
	Two
\end{lstlisting}

\item \textbf{Parity} - тип бита четности, проверка на четность/нечетность или его отсутствие.
\begin{lstlisting}
data Parity =
	Even     |
	Odd      |
	NoParity
\end{lstlisting}

\item \textbf{FlowControl} - флаг включающий, выключающий использование возможности <<притормозить>> удаленное передающее устройство.

\begin{lstlisting}
data FlowControl =
	Software      |
	NoFlowControl
\end{lstlisting}

\item \textbf{SerialPort} - тип, инкапсулирующий открытый COM-порт.

Определенные instances:
\begin{lstlisting}
Typeable   SerialPort	 
BufferedIO SerialPort	 
RawIO      SerialPort	 
IODevice   SerialPort	
\end{lstlisting}

\item \textbf{SerialPortSettings} - настройки COM-порта.
\begin{lstlisting}
data SerialPortSettings =
	commSpeed   :: CommSpeed
	bitsPerWord :: Word8
	stopb       :: StopBits
	parity      :: Parity
	flowControl :: FlowControl
	timeout     :: Int
\end{lstlisting}
Разъяснение параметров конструктора:
\begin{itemize}
\item \textbf{commSpeed} - скорость порта в бодах в секунду;
\item \textbf{bitsPerWord} - количество битов в передаваемых словах (для асихронного режима);
\item \textbf{stopb} - количество стоп битов;
\item \textbf{parity} - тип бита четности;
\item \textbf{flowControl} - наличие контроля потока, возможность <<приостановить>> передающее устройство.
\item \textbf{timeout} - время в десятых долях секунды, после которого подвисшее соединение считается разорванным.
\end{itemize}
\end{itemize}

\subsection{Описание функций}

\begin{lstlisting}
defaultSerialSettings :: SerialPortSettings
\end{lstlisting}
Наиболее часто используемы настройки COM-порта:
\begin{itemize}
\item скорость 9600 бод в секунду;
\item в байте 8 бит;
\item 1 стоп бит;
\item без бита четности;
\item без контроля потока;
\item 0.1 секунда для timeout.
\end{itemize}

\begin{lstlisting}
setSerialSettings :: SerialPort -> SerialPortSettings -> IO SerialPort
\end{lstlisting}
Устанавливает настройки COM-порта:
\begin{itemize}
\item \textbf{SerialPort} - текущий открытый COM-порт;
\item \textbf{SerialPortSettings} - новые настройки COM-порта;
\item \textbf{IO SerialPort} - новое состояние COM-порта.
\end{itemize}

\begin{lstlisting}
getSerialSettings :: SerialPort -> SerialPortSettings
\end{lstlisting}
Получение текущих настроек COM-порта:
\begin{itemize}
\item \textbf{SerialPort} - текущий открытый COM-порт;
\item \textbf{SerialPortSettings} - настройки COM-порта;
\end{itemize}

\begin{lstlisting}
hOpenSerial :: FilePath -> SerialPortSettings -> IO Handle
\end{lstlisting}
Открывает и настраивает COM-порт, возвращает стандартный тип ссылки на устройство ввода/вывода:
\begin{itemize}
\item \textbf{FilePath} - название COM-порта, например <<COM1>> или <<COM2>>;
\item \textbf{SerialPortSettings} - первичные настройки COM-порта;
\item \textbf{Handle} - ссылка на открытый COM-порт;
\end{itemize}

\begin{lstlisting}
openSerial :: FilePath -> SerialPortSettings -> IO SerialPort
\end{lstlisting}
Открывает и настраивает COM-порт:
\begin{itemize}
\item \textbf{FilePath} - название COM-порта, например <<COM1>> или <<COM2>>;
\item \textbf{SerialPortSettings} - первичные настройки COM-порта;
\item \textbf{IO SerialPort} - новый открытый и настроенный COM-порт.
\end{itemize}

\begin{lstlisting}
closeSerial :: SerialPort -> IO ()
\end{lstlisting}
Закрывает открытый COM-порт:
\begin{itemize}
\item \textbf{SerialPort} - открытый COM-порт.
\end{itemize}

\begin{lstlisting}
withSerial :: String -> SerialPortSettings -> (SerialPort -> IO a) -> IO a
\end{lstlisting}
Безопасная функция для работы с COM-портами, закрывает порт после выполенения операции с ним:
\begin{itemize}
\item \textbf{String} - название COM-порта, например <<COM1>> или <<COM2>>;
\item \textbf{SerialPortSettings} - первичные настройки COM-порта;
\item \textbf{SerialPort -> IO a} - функция, которая будет выполнена после открытия порта;
\item \textbf{IO a} - результат выполнения функции над COM-портом.
\end{itemize}

\begin{lstlisting}
send :: SerialPort -> ByteString -> IO Int
\end{lstlisting}
Отсылает байты через COM-порт:
\begin{itemize}
\item \textbf{SerialPort} - текущий открытый COM-порт;
\item \textbf{ByteString} - буфер для отсылки;
\item \textbf{Int} - количество байтов, которое было отослано.
\end{itemize}

\begin{lstlisting}
recv :: SerialPort -> Int -> IO ByteString
\end{lstlisting}
Получение байтов из COM-порта с ограничением по количеству байтов сверху:
\begin{itemize}
\item \textbf{SerialPort} - текущий открытый COM-порт;
\item \textbf{Int} - максимальное количество байтов для приема;
\item \textbf{IO ByteString} - принятый буфер.
\end{itemize}

\begin{lstlisting}
flush :: SerialPort -> IO ()
\end{lstlisting}
Принудительно отсылает ждущие отправки данные во временных буфера COM-порта:
\begin{itemize}
\item \textbf{SerialPort} - текущий открытый COM-порт.
\end{itemize}

\begin{lstlisting}
setDTR :: SerialPort -> Bool -> IO ()
\end{lstlisting}
Устанавливает сигнал \textbf{DTR} (Data Terminal Ready) в заданное значение:
\begin{itemize}
\item \textbf{SerialPort} - текущий открытый COM-порт;
\item \textbf{Bool} - значение, которое должен принять \textbf{DTR} провод.
\end{itemize}

\begin{lstlisting}
setRTS :: SerialPort -> Bool -> IO ()
\end{lstlisting}
Устанавливает сигнал \textbf{RTS} (Ready To Send) в заданное значение:
\begin{itemize}
\item \textbf{SerialPort} - текущий открытый COM-порт;
\item \textbf{Bool} - значение, которое должен принять \textbf{RTS} провод.
\end{itemize}

\section{Канальный уровень}
\subsection{Функции канального уровня}
Канальный уровень выполняет следующие функции:
\begin{enumerate}
\item Запрос логического соединения;
\item Разбивка данных на блоки (кадры);
\item Управление передачей кадров;
\item Обеспечение необходимой последовательности блоков данных, передаваемых через межуровневый интерфейс;
\item Контроль и обработка ошибок;
\item Проверка поддержания соединения;
\item Запрос на разъединение логического соединения.
\end{enumerate}

\subsection{Протокол связи}
В основном протокол содержит набор соглашений или правил, которого должны придерживаться обе стороны связи для обеспечения получения и корректной интерпретации информации, передаваемой между двумя сторонами. Таким образом, помимо управления ошибками и потокм протокол связи регулирует также такие вопрсы, как формат передаваемых данных - число битов на каждый элемент и тип используемой схемы кодирования, тип и порядок сообщений, подлежащих обмену для обеспечения (свободной от ошибок и дубликатов) передачи информации между двумя взаимодействующими сторонами.

Перед началом передачи данных требуется установить соединение между двумя сторонами, тем самым проверяется доступность приемного устройства и его готовность воспринимать данные. Для этого передающее устройство посылает специальную команду: запрос на соединение, сопровождаемую ответом приемного устройства, например о приеме или отклонении вызова.

Также необходимо информировать пользователя о неисправностях в физическом канале, поэтому для поддержания логического соединения необходимо предусмотреть специальный кадр, который непрерывно будет посылаться с одного компьютера на другой, сигнализируя тем самым, что логическое соединение активно.

\subsection{Защита передаваемой информации}
\end{document}