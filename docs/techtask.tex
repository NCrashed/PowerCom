\documentclass[14pt]{extarticle}

\usepackage[14pt]{extsizes}
\usepackage{amssymb}
\usepackage{amsmath}

\usepackage{graphicx}

\usepackage{mathtext} % Русские буквы в формулах
\usepackage{array}

\usepackage{longtable}
\usepackage[left=2cm,right=2cm,
    top=2cm,bottom=2cm,bindingoffset=0cm]{geometry}
    
\usepackage{latexsym} %Включает latex пакеты
\usepackage[T2A]{fontenc} %Подключает поддержку кодировок, отличных от латиницы.
\usepackage[utf8]{inputenc} %Указывает, что текст будет вводится в KOI8-R кодировке.
\usepackage[russian]{babel} %Подключает интернациональный паке babel.
\usepackage{setspace}
\pagestyle{plain} \onehalfspacing

\usepackage{ccaption}
\captiondelim{. }

\begin{document}

\begin{singlespace}
\thispagestyle{empty}
\begin{center}
МОСКОВСКИЙ ГОСУДАРСТВЕННЫЙ УНИВЕРСИТЕТ\\ ИМЕНИ Н.~Э.~БАУМАНА\\[0.5cm]
\end{center}


\vspace{4.5cm}

\begin{center}
\textbf{
\LARGE{Техническое задание}\\
<<Локальная безадаптерная сеть>>\\
по курсу <<Сетевые технологии>> \\[0.5cm]
}
\end{center}

\vspace{0.7cm}

\begin{flushright}
\begin{minipage}{6cm}
Утверждаю\\
\underline{\hspace{2cm}} Галкин В.А.\\
"\underline{\hspace{0.5cm}}"\underline{\hspace{1.5cm}} 2013 г.\\[0.5cm]
\end{minipage}
\end{flushright}


\begin{flushright}
\begin{minipage}{6cm}
Исполнители:\\
Гуща А.В.\\
студент группы ИУ5-72\\ [0.2cm] 

Нардид А.Н. \\
студент группы ИУ5-72\\ [0.2cm]

Оганян Л.П. \\
студент группы ИУ5-72\\ [0.2cm]
\end{minipage}
\end{flushright}

\vspace{2.5cm}

\begin{center}
Москва 2013 г.
\end{center}

\end{singlespace}

\pagebreak
\thispagestyle{empty}
\tableofcontents
\clearpage

\thispagestyle{plain}
\section{Наименование}
Программа пересылки текстовых сообщений <<PowerCom>>.

\section{Основание для разработки}
Основанием для разработки является учебный план МГТУ им. Баумана кафедры ИУ5 на 7 семестр.

\section{Исполнители}
Исполнителями являются студенты МГТУ им. Н.Э.Баумана группы ИУ5-72: Оганян Л.П. (пользовательский уровень), Гуща А.В. (канальный уровень), Нардид А.Н. (физический уровень).

\section{Цель разработки}
Разработать протоколы взаимодействия объектов до прикладного уровня локальной сети, состоящей из двух ПЭВМ, соединенных нульмодемно через интерфейс RS-232C, и реализующией цункцию передачи текста диалога абонентов. Принимаемый и передаваемый тексты отображать в разных окнах. Скорость обмена и параметры COM-порта выбираются пользователем. Передаваемую информацию защитить циклическим кодом $X^3 + X + 1$.

\clearpage
\section{Содержание работы}
\subsection{Задачи, подлежащие решению}
\begin{itemize}
\item Разработать протоколы взаимодействия объектов прикладного, канального и физического уровней локальной сети;
\item Защитить передаваемую информацию циклическим кодом;
\item Реализовать функцию передачи текстовых сообщенией между двумя ПЭВМ.
\end{itemize}

\subsection{Требования к программному изделию}
\subsubsection{Требования к функциональным характеристикам}
Программа должна контролировать процессы, связанные с получением, использованием и освобождением различных ресурсов ПЭВМ. При возникновении ошибок обрабатывать их, а в случае необходимости:
\begin{itemize}
\item извещать пользователя своей ПЭВМ;
\item извещать ПЭВМ на другом конце канала.
\end{itemize}

Номер COM-порта и скорость передачи по каналу устанавливается через меню.

\subsubsection{Функции физического уровня}
\begin{itemize}
\item Установление параметров COM-порта;
\item Установление, поддержание и разъединение физического канала.
\end{itemize}

\subsubsection{Функции канального уровня}
\begin{itemize}
\item Запрос логического соединения;
\item Управление передачей кадров;
\item Обеспечение необходимой последовательности блоков данных, передаваемых через межуровневый интерфейс;
\item Контроль и исправление ошибок;
\item Запрос на разъединение логического соединения.
\end{itemize}

\subsubsection{Функции пользовательского уровня}
\begin{itemize}
\item Интерфейс с пользователем через систему меню;
\item Установка режима работы;
\item Установка номера COM-порта для канала;
\item Установка скорости передачи;
\item Отображение диалога пользователей в разных окнах.
\end{itemize}

\subsection{Входные и выходные данные}
\subsubsection{Входные данные}
\begin{itemize}
\item Текст сообщения, вводимый с клавиатуры передающей ПЭВМ.
\end{itemize}

\subsubsection{Выходные данные}
\begin{itemize}
\item Принятый текст сообщения на экране принимающей ПЭВМ.
\end{itemize}
\clearpage

\section{Требования к составу технических средств}
Программное изделие выполняется на языке программирования Haskell под управлением операционной системы Microsoft Windows XP/Microsoft Windows 7 и под управлением операционной системы GNU/Linux.

Для работы требуется две ПЭВМ типа IBM PC AT (/XT), соединенные нульмодемным кабелем через интерфейс RS-232C.

Программа должна распространяться под лицензией GNU General Public License v3 (GNU GPLv3).

\section{Этапы разработки}
\begin{enumerate}
\item Разработка Технического Задания, до 15.09.13;
\item Разработка Эскизного Проекта до 25.09.13;
\item Разработка Технического Проекта до 30.10.13;
\item Разработка Программы до 20.12.13.
\end{enumerate}

\section{Техническая документация, предъявляемая по окончанию работы}
\begin{enumerate}
\item Техническое Задание;
\item Технический проект:
	\begin{itemize}
	\item Расчетно-пояснительная записка;
	\item Комплект технической документации на программный продукт, включающий:
		\begin{itemize}
		\item описание программы;
		\item руководство пользователя;
		\item программу и методику испытания.
		\end{itemize}

	\item Графическая часть на 3 (6) листах формата А1(А2):
		\begin{itemize}
		\item Структурная схема программы;
		\item Структурная протокольных блоков данных;
		\item Структурные схемы основных процедур взаимодействия программных сущностей по разработанным протоколам;
		\item Временные диаграммы работы протоколов;
		\item Граф диалога пользователя;
		\item Алгориты программ.
		\end{itemize}

	\item Электронный носитель информации с технической и программной документацией.
	\end{itemize}
\end{enumerate}

\section{Порядок приемки работы}
Приемка работы осуществляется в соответствии с <<Программой и методикой испытаний>>. Работа защищается перед комиссией преподавателей кафедры.

\section{Дополнительные условия}
Данное Техническое Задание может дополняться и изменяться в установленном порядке.
\end{document}